\documentclass[spanish]{article}
\usepackage[spanish]{babel}
\usepackage[utf8]{inputenc}
\usepackage[colorlinks]{hyperref}
\usepackage{fourier}
\usepackage{amssymb, amsmath, tikz}
\usepackage{listings}
\usepackage{float}
\usepackage[top = 2.5cm, bottom = 2cm, left = 2cm, right = 2cm]{geometry}
\usepackage{color}
\usepackage{hyperref}

\title{\begin{center}
  \label{fig:ejemplo}
 \end{center}
  Tarea 1\\ Multigrid\\
  \textit{}
}

\author{
  Camilo Valenzuela Carrasco \\ \texttt{camilo.valenzuela@alumnos.usm.cl}
  \\ 201173030-5
}

\lstset{
  language=Python
}

\newcommand{\multlinecomment}[1]{\directlua{-- #1}}


\begin{document}
  \maketitle
  %\section{Introducción}
  
    
  %\section{Preguntas}
  
  
  %PREGUNTAS
  
  \begin{enumerate}
      %Pregunta 1
      \item Considere el intervalo $0 \leq x \leq 1$ con nodos $x_j =\frac{j}{n}$, $0 \leq n \leq n$. Muestre que el k-ésimo modo de Fourier $w_{k,j} = sin\frac{jk\pi}{n}$ tiene una longitud de onda $l = \frac{2}{k}$ ¿Cuál modo tiene longitud de onda $l = 8h$? ¿y $l = \frac{1}{4}$?
      
      \textbf{Solución:} 
      
      Si tenemos $sin(\alpha x)$ la longitud de onda esta dada por $l = \frac{2\pi}{\alpha}$. Sabiendo esto podemos transformar 
      
      $$sin \left(\frac{jk\pi}{n}\right) = sin \left(\frac{j}{n} k\pi \right) = sin\left(x_j k\pi\right)$$
      
      Con $x = x_j$ y $\alpha = k\pi$ la longitud de onda queda:
      
      $$ l = \frac{2\pi}{k\pi} = \frac{2}{k} \ \ \ \square$$
      
      Reemplazando $l = 8h$ y $l = \frac{1}{4}$ obtenemos $k = \frac{1}{4h}$ y $k=8$ respectivamente.
      
      \item El número de condición de una matriz $\kappa(A) = ||A||_2 ||A^{-1}||_2$ da una idea de cuan bien el residual mide al error. En el siguiente ejercicio, use la propiedad que $||Ax||\leq ||A||||x||$
      
        %SUBPREGUNTAS 1)
        \begin{enumerate}
            \item Partiendo de la relación $Ae = r$ y $A^{-1}f = u$, demuestre que
            
            $$ \frac{||r||_2}{||f||_2} \leq \kappa(A) \frac{||e||_2}{||u||_2}$$
            
           Entregue una interpretación de esta desigualdad cuando $\kappa$ es alto.
            
            \textbf{Solución:}
            
            Comenzamos aplicando la norma 2 a ambas ecuaciones
            $$Ae = r$$
            $$||Ae||_2 = ||r||_2$$
            
            $$    A^{-1}f = u $$
            $$    ||A^{-1}f||_2 = ||u||_2 $$
            
            Multimplicamos 
            $$  ||Ae||_2 ||u||_2 = ||r||_2  ||u||_2 $$
            $$ ||Ae||_2 ||A^{-1}f||_2 = ||r||_2  ||u||_2  $$
            
            Utilizando la desigualdad triangular obtenemos
           $$ ||A||_2 ||A^{-1}||_2 ||e||_2 ||f||_2 \geq ||Ae||_2 ||A^{-1}f||_2 = ||r||_2  ||u||_2 $$ 
           $$ ||A||_2 ||A^{-1}||_2 ||e||_2 ||f||_2 \geq ||r||_2  ||u||_2 $$
           Reemplazando $\kappa(A)$ y ordenando la ecuación obtenemos
           
           $$ \kappa(A) ||e||_2 ||f||_2 \geq ||r||_2  ||u||_2 $$
           $$  \frac{||r||_2}{||f||_2}\leq \kappa(A)\frac{ ||e||_2 }{ ||u||_2 } \ \ \  \square $$
        
           Mientras mayor sea el valor de $\kappa$ el residuo de la relajación puede ser mucho más grande, por lo que nuestra
           solución aproximada por la relajación puede quedar muy lejos de la solución exacta.
           
           Lo que nos da una cota inferior para el error normalizado.
        
           \item Partiendo de las relaciones $Au = f$ y $A^{-1}r = e$ demuestre que 
           $$\frac{||e||_2}{||u||_2} \leq \kappa(A) \frac{||r||_2}{||f||_2}$$
           
           Entregue una interpretación de esta desigualdad cuando $\kappa$ es alto.
    
        \textbf{Solución:}
        
        Utilizando la misma idea que en el ejercicio anterior:
        
        $$||Au||_2 = ||f||_2 \ \ \  , \ \ \ ||A^{-1}r||_2 = ||e||_2 $$
        
        Multiplicando $||e||_2$ a la primera ecuación
        
        $$||Au||_2 ||e||_2 = ||f||_2 ||e||_2 $$
        $$ ||Au||_2 ||A^{-1}r||_2 = ||f||_2 ||e||_2 $$
        
        Utilizando la desigualdad triangular
        
        $$ ||A||_2 ||A^{-1}||_2 ||u||_2  ||r||_2 \geq ||Au||_2 ||A^{-1}r||_2 = ||f||_2 ||e||_2 $$
        $$ \kappa(A) ||u||_2  ||r||_2 \geq ||f||_2 ||e||_2 $$
        
        Despejando obtenemos la desigualdad buscada
         $$\frac{||e||_2}{||u||_2} \leq \kappa(A) \frac{||r||_2}{||f||_2}  \ \ \  \square$$
         
         Lo mismo que el caso anterior pero ahora tenemos una cota superior del error normalizado.
        
        \item Combine estos resultados para llegar a
        
        
        $$ \frac{1}{\kappa(A)} \frac{||r||_2}{||f||_2} \leq \frac{||e||_2}{||u||_2} \leq \kappa(A) \frac{||r||_2}{||f||_2}$$
        
        
        \textbf{Solución:}
 
        Despejamos $\frac{||e||_2}{||u||_2}$ en ambas ecuaciones
        
        $$ \frac{1}{\kappa(A)} \frac{||r||_2}{||f||_2} \leq \frac{||e||_2}{||u||_2} \ \ \ \ , \ \ \ \  \frac{||e||_2}{||u||_2} \leq \kappa(A) \frac{||r||_2}{||f||_2}$$
        
        Al combinar ambas desigualdades se obtienen las dos cotas para el error normalizado.
        
        $$ \frac{1}{\kappa(A)} \frac{||r||_2}{||f||_2} \leq \frac{||e||_2}{||u||_2} \leq \kappa(A) \frac{||r||_2}{||f||_2}$$
        
        Mientra mayor es el valor de $\kappa$, mayor va a ser el rango en el que se puede mover el error normalizado, lo que hace que sea dificil la convergencia de la relajación.
        
        
        \end{enumerate}
      
      %Pregunta 2
       \item Considere un método iterativo del tipo $v = v + B^{-1}(f - Av)$ aplicado al problema $Au = f$. Use los siguientes pasos para demostrar que la relajación de $Au = f$ es equivalente a relajar $Ae = r$ con estimación inicial 0.
       
       \begin{itemize}
           \item Considere el problema $Au = f$ con una estimación inicial arbitraria $v = v_0$ ¿Cuál es el error y residual asociado con $v_0$?
            \item Ahora considere la ecuación del residual $Ae = r0 = f - Av_0$. ¿Cuál es el error y residual en la estimación inicial $e_0$?
            \item Concluya que ambos problemas son equivalentes.
       \end{itemize}
       
       
       \textbf{Solución:}
       
       \begin{itemize}
                \item Dado $Au = f$ con un vector inicial arbitrario $v = v_0$ se tiene que el error inicial asociado es $e = u - v_0$ y el residuo $r = f - Av_0$
                \item Con $Ae = r$ tenemos algo similar, donde el error está dado por $\hat{e} = e - e_0$  y el residuo por $\hat{r} = r_0 - Ae_0$. Como $e_0 = 0$ tenemos
                \begin{center}
                    $ \hat{e} = e = u - v_0 $ y $\hat{r} = r_0 = f - Av_0$
                \end{center}
                \item Como el error inicial y el residuo inicial en ambos casos es el mismo, se puede decir que ambas relajaciones son equivalentes.
       \end{itemize}
       
 
%       \begin{itemize}
%           \item Comenzando con un $v_0$ arbitrario y relajando $Au = f$ con el método iterativo tenemos
%           $$ v^{k+1} = v^{k} + B^{-1} ( f - A v^k)$$
%           
%           Se tiene que $r = f - Av $ despejamos $v$
%           $$Av = r - f $$
%           $$ v = A^{-1} (f - r) $$  
%
%          Reemplazamos en nuestro método iterativo
%          
%          $$ A^{-1} (f - r^{k+1}) =  A^{-1} (f - r^{k}) + B^{-1} (f - Av^k)$$
%          
%          Simplificamos algunos términos y se obtiene
%          
%          $$ - A^{-1} r^{k+1} =  - A^{-1} r^{k} + B^{-1} (f - Av^k) $$ 
%
%        Como $r^k=f - Av^k$ reemplazamos y cambiamos los signos
%        \begin{equation}
%            \label{eq:con_a}  A^{-1} r^{k+1} =   A^{-1} r^{k} - B^{-1} r^k 
%        \end{equation}
%  
%        Multiplicando por A a la izquierda obtenemos el residuo en cada iteración.
%        \begin{center}
%        \boxed{r^{k+1} =   r^{k} - AB^{-1} r^k }
%        \end{center}
%        Desde la ecuación (\ref{eq:con_a}) usando $Ae = r$, $e = A^{-1}r$ obtenemos el error en la iteración.
%        
%        \begin{center}
%            \boxed{e^{k+1} = e^{k} - B^{-1} r^k}
%        \end{center}
%            
%       \end{itemize}

      
    
        \item Verifique que $I_h^{2h} w^h_k = cos(\frac{k\pi}{2n})w_k^{2h}$ donde $w_{k,j}^h = sin(\frac{jk\pi}{n}$, $1 \leq k \leq \frac{n}{2}$ e $I_h^{2h}$ es el operador \textit{full weightning}
        
        
        \textbf{Solución:}
        
        Por el operador  $I_h^{2h}$ se tiene que:
        $$I_h^{2h} w_{j,k}^{2h} =\frac{1}{4}\left( w_{2j-1,k}^h + 2  w_{2j,k}^h +  w_{2j+1,k}^h \right) $$
        
        Reemplazando los $w_{k,j}^h$
        
        $$I_h^{2h} w_{j,k}^{2h} = 
            \frac{1}{4}\left( 
                sin \left(\frac{(2j-1)k\pi}{n}\right)
                + 2 sin \left(\frac{(2j)k\pi}{n} \right)
                + sin \left(\frac{(2j+1)k\pi}{n} \right)
                \right)
        $$
        
        Utilizando la identidad trigonométrica $sin(a+b) = sin(a) cos(b) + sin(b) cos (a)$
        $$I_h^{2h} w_{j,k}^{2h} = 
            \frac{1}{4}\left( 
                sin \left(\frac{2jk\pi}{n}\right) cos\left(\frac{k\pi}{n}\right)
                - sin \left(\frac{k\pi}{n}\right)  \left(cos\frac{2jk\pi}{n} \right)
                + 2 sin \left(\frac{2jk\pi}{n} \right)
                + sin \left(\frac{2jk\pi}{n}\right) \left(cos\frac{k\pi}{n} \right)
                + sin \left(\frac{k\pi}{n}\right) \left(cos\frac{2jk\pi}{n}  \right)
                \right)
        $$
        
        Como $cos\frac{2jk\pi}{n} = 0$ 
        
        $$I_h^{2h} w_{j,k}^{2h} = 
            \frac{1}{4}\left( 
                sin \left(\frac{2jk\pi}{n}\right)  \left(cos\frac{k\pi}{n}\right)
                + 2 sin \left(\frac{2jk\pi}{n}\right)
                + sin \left(\frac{2jk\pi}{n}\right)  \left(cos\frac{k\pi}{n}\right)
                \right)
        $$
        
        $$I_h^{2h} w_{j,k}^{2h} = 
            \frac{1}{4}\left( 
                2 sin\left(\frac{2jk\pi}{n} cos\frac{k\pi}{n}\right)
                + 2 sin\left(\frac{2jk\pi}{n}\right)
            \right)
        $$
        
        Factorizando por $2 \ sin\left(\frac{2jk\pi}{n}\right)$
        $$I_h^{2h} w_{j,k}^{2h} = 
            \frac{1}{2} sin\left(\frac{2jk\pi}{n}\right) \left( 
                cos\left(\frac{k\pi}{n}\right)
                + 1
            \right)
        $$

        Utilizando la identidad trigonométrica $cos^2(a) = \frac{1 + cos(2a)}{2}$ con $a =\frac{k\pi}{2n}$ 
        
        $$I_h^{2h} w_{j,k}^{2h} = cos^2 \left(\frac{k \pi}{2n}\right) sin\left(\frac{2jk\pi}{n}\right)$$
        
        Como $ sin\left(\frac{2jk\pi}{n}\right) =  sin\left(\frac{2jk\pi}{\frac{n}{2}}\right) = w_{j,k}^{2h}$
        
         $$I_h^{2h} w_{j,k}^{2h} = cos^2 \left(\frac{k \pi}{2n}\right) w_{j,k}^{2h}$$
         
         Y en general se tiene 
         
           $$I_h^{2h} w_{k}^{2h} = cos^2 \left(\frac{k \pi}{2n}\right) w_{k}^{2h} \ \ \ \square$$
  \end{enumerate}
  

\end{document}